\documentclass[12pt]{article}

% Автор: Сергей Копелиович

\usepackage{polyglossia}

\usepackage{amssymb}
\usepackage[russian]{hyperref}
\usepackage{datetime}
\usepackage{cmap}
\usepackage{enumerate}
\usepackage{hologo}
\usepackage{minted}
\usepackage{unicode-math}
\usepackage{lastpage}

\defaultfontfeatures{Ligatures=TeX}
\setmainfont{CMU Serif}  
\setmonofont{CMU Typewriter Text}  
\setsansfont[Mapping=tex-text]{CMU Sans Serif}
\setmathfont{Latin Modern Math}
\setdefaultlanguage[spelling=modern]{russian}
\setotherlanguage{english}

\sloppy

\voffset=-25mm
\textheight=230mm
\hoffset=-25mm
\textwidth=180mm

\def\EPS{\varepsilon}
\def\SO{\Rightarrow}
\def\EQ{\Leftrightarrow}
\def\t{\texttt}
\def\O{\mathcal{O}}
\newcommand{\q}[1]{\langle #1 \rangle} % <x>
\newcommand\URL[1]{{\footnotesize{\url{#1}}}}
\def\myindent{\hspace*{\parindent}}
\def\up{\vspace*{-\baselineskip}}
\def\NO{\t{\#}}
\newcommand{\ITEM}[1]{{\bf \underline{#1}}\hspace{0.5em}}

\makeatletter

\renewcommand{\@oddfoot}{
    \parbox{\textwidth}{
        \sffamily
        {{\hfil}\thepage/\pageref*{LastPage} \hfil}%
    }
}

\newenvironment{MyList}{
  \begin{enumerate}
  \setlength{\parskip}{-5pt}
  \setlength{\itemsep}{5pt}
}{
  \end{enumerate}
}

\newcommand\Section[1]{\vspace*{-1em}\section{#1}}
\begin{document}

\begin{center}
  {\Large \bf CS-Club, осенний семестр 2014, курс алгоритмов} \\ 
  \vspace{0.5em}
  {\Large \bf Реализация бинарной кучи} \\
  \vspace{0.5em}
  {\large Сергей Копелиович} \\
  \vspace{0.5em}
  {Собрано {\today} в {\currenttime}}
\end{center}

\vspace{-1em}
\noindent \underline{\hbox to 1\textwidth{{ } \hfil{ } \hfil{ } }}

\Section{Постановка задачи}

Реализовать бинарную кучу. \\
Протестировать корректность реализации. \\
Протестировать скорость работы. \\
Сравнить с \t{set<int>}, \t{priority\_queue<int>}
              
\Section{Реализация}

Реализован следующий основной интерфейс:
\begin{minted}{c}
  void build( int an, POINTER a ); // O(n)
  void add( int x ); // O(log n)
  T extractMin(); // O(logn)
  T getMin(); O(1)
\end{minted}
И две внутренние дополнительные функции:
\begin{minted}{c}
  void siftUp( int i ); // O(log i)
  int siftDown( int i ); // O(log(n/i))
\end{minted}

\Section{Тестирование}

\myindent{}Тест \NO{1}: 
генерируем случайный массив из $n$ чисел, строим по нему кучу, достаем по очереди элеметы из кучи.
Замеряем время работы. 
Сравниваем результат работы решения с результатом наивного решения, вынимающего минимумы из массива за $\O(n)$. 

Тест \NO{2}:  
генерируем случайный массив из $n$ чисел, добавляем по одному элементы в кучу, достаем по очереди элеметы из кучи.
Сравниваем результат работы решения с результатом наивного решения, вынимающего минимумы из массива за $\O(n)$. 

Чтобы воспроизвести тесты, запустите \t{bash run.sh}

\Section{Результаты тестирования}

H -- наша реализация бинарной кучи \\
Q -- \t{priority\_queue<int>} \\
S -- \t{multiset<int>}

\vspace{1em}
\begin{tabular}{|c|r|r|r|}
\hline
n/algo & H & Q & S \\
\hline
$10^6$ & 0.61 sec & 0.50 sec & 1.64 sec \\
\hline
$10^7$ & 11.14 sec & 10.13 sec& 26.82 sec \\
\hline
\end{tabular}

\end{document}
