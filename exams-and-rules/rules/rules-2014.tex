\documentclass[12pt]{article}

% Автор: Сергей Копелиович

\usepackage{polyglossia}

\usepackage{amssymb}
\usepackage[russian]{hyperref}
\usepackage{datetime}
\usepackage{cmap}
\usepackage{enumerate}
\usepackage{hologo}
\usepackage{minted}
\usepackage{unicode-math}
\usepackage{lastpage}

\defaultfontfeatures{Ligatures=TeX}
\setmainfont{CMU Serif}  
\setmonofont{CMU Typewriter Text}  
\setsansfont[Mapping=tex-text]{CMU Sans Serif}
\setmathfont{Latin Modern Math}
\setdefaultlanguage[spelling=modern]{russian}
\setotherlanguage{english}

\sloppy

\voffset=-30mm
\textheight=240mm
\hoffset=-25mm
\textwidth=180mm

\def\EPS{\varepsilon}
\def\SO{\Rightarrow}
\def\EQ{\Leftrightarrow}
\def\t{\texttt}
\def\O{\mathcal{O}}
\newcommand{\q}[1]{\langle #1 \rangle} % <x>
\newcommand\URL[1]{{\footnotesize{\url{#1}}}}
\def\myindent{\hspace*{\parindent}}
\def\up{\vspace*{-\baselineskip}}
\def\NO{\t{\#}}
\newcommand{\ITEM}[1]{{\bf \underline{#1}}\hspace{0.5em}}

\makeatletter

\renewcommand{\@oddfoot}{
    \parbox{\textwidth}{
        \sffamily
        {{\hfil}\thepage/\pageref*{LastPage} \hfil}%
    }
}

\newenvironment{MyList}{
  \begin{enumerate}
  \setlength{\parskip}{-5pt}
  \setlength{\itemsep}{5pt}
}{
  \vspace*{-1em}
  \end{enumerate}
}

\newcommand\Section[1]{\vspace*{-1em}\section{#1}}
\begin{document}

\begin{center}
  {\Large \bf CS-Club, осенний семестр 2014, курс алгоритмов} \\ 
  \vspace{0.5em}
  {\Large \bf Правила получения зачета} \\
\end{center}

\vspace{-1em}
\noindent \underline{\hbox to 1\textwidth{{ } \hfil{ } \hfil{ } }}

\Section{Что нужно сделать?}

\myindent{}Выбрать оценку, на которую вы претендуете и сделать {\bf\it любое одно} из заданий на эту оценку.
Во всех заданиях нужно реализовать предложенный алгоритм
или структуру данных и сделать для него тесты.
В качестве результата нужно предоставить
исходный код и краткий отчет с результатами тестов.

\vspace*{0.5em}
Список задач будет пополняться по ходу курса.

\Section{Как сдать?}

Прислать на почту архив с файлами. В адресе указать Павла Маврина \url{pavel.mavrin@gmail.com}
и Сергея Копелиовича \url{burunduk30@gmail.com}. Обязательно указать обоих.
В теме письма добавить префикс \t{[cs-club]}.

\Section{Пример готового задания}

Тема: реализовать бинарную кучу. \\
Ссылка: \url{http://acm.math.spbu.ru/~sk1/mm/cs-club/sample-report-2014.7z}.
              
\Section{Задания на 3}
\begin{MyList}
\item Любое функциональное сбалансированное дерево поиска (path cloning)
\item Orthogonal Range Query: статическая задача в 2D c $\O(\log^2 n)$ на запрос
\item Fractional Cascading: фреймворк для Fractional Cascading на списке
\item External Memory: стек и очередь
\item Cache oblivious: транспонирование матрицы
\item Inplace stable sort за $\O(n \log^2 n)$
\item Pairing Heap
\item Dynamic 2-Edge-Connectivity в offline, ребра только добавляются за $\O((n+m)\log (n+m))$
\end{MyList}
\Section{Задания на 4}
\begin{MyList}
\item Частично сбалансированный связный список с $\O(1)$ на операцию (fat nodes + cloning)
\item Orthogonal Range Query: статическая задача в 2D c $\O(\log n)$ на запрос
\item Orthogonal Range Query: статическая задача в $R^d$ c $O(\log^d n)$ на запрос
\item Fractional Cascading: фреймворк для Fractional Cascading на дереве
\item External Memory: сортировка ($k$-блочный merge sort)
\item External Memory: B-дерево
\item Cache oblivious: дерево интервалов 
\item Inplace stable merge за $\O(n + m^{1 + \EPS})$
\item Merge за $\O(n)$ (не обязательно стабильный)
\item Weak Heap (слабая куча)
\item MinMax Heap (inplace куча, умеющая доставить и минимум, и максимум)
\item Dynamic Connectivity в offline за $\O((n+m)\log m)$ или $\O((n+m)\sqrt{n})$
\end{MyList}
\Section{Задания на 5}
\begin{MyList}
\item Частично сбалансированное красно-черное дерево с $\O(1)$ памяти на операцию (fat nodes + cloning)
\item Orthogonal Range Query: статическая задача в $R^d$ c $\O(\log^{d-1}n)$ на запрос
\item Orthogonal Range Query: динамическая задача в 2D c $\O(\log^2 n)$ на запрос
\item Fractional Cascading: фреймворк для Fractional Cascading на ациклическом графе с ограничениями на ребрах
\item External Memory: разворачивание списка за $\O(n)$
\item External Memory: куча
\item Leftist Heap, Skew Heap. К обеим кучам добавить insert за $\O(1)$ и merge за $\O(1)$ (bootstrapping).
\item Binomial Heap (биномиальная куча), Fibonacci Heap (куча фибоначчи)
\item Дейкстра с Radix Heap за $\O(m + n \log C)$.
\item Dynamic 2-Edge-Connectivity в offline за $\O((n+m)\log m)$
\end{MyList}
\Section{Задания на 5+}
\begin{MyList}
\item Orthogonal Range Query: динамическая задача в 3D c $\O(\log n)$ на запрос
\item Inplace stable sort за $\O(n)$.
\item Дейкстра с двухуровенвой Radix Heap с Fibonacci Heap за $\O(m + n\sqrt{\log C})$
\end{MyList}

\Section{Теорзачет}

Последняя пара состояится 14-го декабря.
20-го декабря (суббота) у вас будет возможность прийти к
??:?? в ПОМИ и поучаствовать в <<теорзачете>>.
Смысл мероприятия -- освежить знания теории.

\end{document}
